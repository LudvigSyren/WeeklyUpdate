%++++++++++++++++++++++++++++++++++++++++
% Don't modify this section unless you know what you're doing!
\documentclass[letterpaper,12pt]{article}
\usepackage{tabularx} % extra features for tabular environment
\usepackage{amsmath}  % improve math presentation
\usepackage{graphicx} % takes care of graphic including machinery
\usepackage[margin=1in,letterpaper]{geometry} % decreases margins
\usepackage{cite} % takes care of citations
\usepackage[final]{hyperref} % adds hyper links inside the generated pdf file
\hypersetup{
	colorlinks=true,       % false: boxed links; true: colored links
	linkcolor=blue,        % color of internal links
	citecolor=blue,        % color of links to bibliography
	filecolor=magenta,     % color of file links
	urlcolor=blue         
}
%++++++++++++++++++++++++++++++++++++++++


\begin{document}

\title{Constraint Modelling}
\author{Ludvig Syrén and Christoffer Fjellstedt}
\date{\today}
\maketitle

\section{Proposed model}
An elastic beam is divided into two rigid bodies which relative motion among each
other is constrained at all six dof with a compliance joint. The constraint on 
the translational movement is formulated similair to a ball joint \cite{Erleben}
and the constraint on the roational movement is formulated as the 
"angular constraint" in \cite{IEEE}.

\subsection{Translational}
Consider two local reference frame attached to body $1$ and $2$, 
$\vec{e^1_x}, \vec{e^1_y}, \vec{e^1_z}$ and $\vec{e^2}_x, \vec{e^2_y}, \vec{e^2_z}$,
 in a global reference frame $\vec{e_x}, \vec{e_y}, \vec{e_z}$. The distance between
 body $1$ and body $2$ is a vector in the global reference frame which is initially a
 constant length in the local reference system of body $1$
 \begin{equation}
    d^0_{1,2}=
    \begin{pmatrix}
        \Delta x^0\\
        \Delta y^0\\
        \Delta z^0.
    \end{pmatrix} 
 \end{equation}
 Where $\Delta x^0$ is the intial length in $e^1_x$ and so on.
At time $t>0$ the distance is expressed as a vector in the global reference, $d_{1,2}$ system 
projected onto the local reference system of body $1$
\begin{equation}
    d^1_{1,2}=R(e_1)d_{1,2}
\end{equation}
where $R(e_1)$ is the translation matrix of local reference system $1$ expressed in terms
of euler angles $e_1$. Since we want the two bodies to act as one the three translational constraint
is then\footnote{Might as well use ball-joint?}
\begin{equation}
    d^1_{1,2}-d^0_{1,2}=0.
\end{equation}

\subsection{Rotational}
We want to constrain the bending and twisting between body $1$ and $2$ to do this we again
consider the two local reference frames of body $1$ and $2$,
$\vec{e^1_x}, \vec{e^1_y}, \vec{e^1_z}$ and $\vec{e^2}_x, \vec{e^2_y}, \vec{e^2_z}$. Say 
that the bending occurs around $\vec{e^1_x}, \vec{e^2}_x$, the cross-product of
\begin{equation}
    \label{cross_x}
    \vec{e^1_y} \times \vec{e^2_y}=|\vec{e^1_y}| |\vec{e^2_y}| sin(\theta) \vec{n}
\end{equation}
and the dot product of
\begin{equation}
    \label{dot_x}
    \vec{e^1_y} \cdot \vec{e^2_y}=|\vec{e^1_y}| |\vec{e^2_y}| cos(\theta)
\end{equation}
where $\theta$ is the angle in the plane of $\vec{e^1_y}, \vec{e^1_z}$ and
 $\vec{e^2_y}, \vec{e^2_z}$\footnote{Is this correct?}. Combining equation \eqref{cross_x} and \eqref{dot_x}
 we find that 
 \begin{equation}
     \frac{|\vec{e^1_y} \times \vec{e^2_y|}}{\vec{e^1_y} \cdot \vec{e^2_y}}=tan(\theta)
 \end{equation}
which is valid for angles in between $-\frac{\pi}{2}< \theta < \frac{\pi}{2}$ and 
the constraint on bending motion is $\theta=0$.\\
\\
Say that the twisting occurs around $\vec{e^1_y}, \vec{e^2_y}$ then with the same approach
as for the bending constraint, we find the angle $\phi$ in the plane of $\vec{e^1_x},
 \vec{e^1_z}$ and $\vec{e^2_x}, \vec{e^2_z}$\footnote{Is this correct?} from 
\begin{equation}
    \frac{|\vec{e^1_z} \times \vec{e^2_z|}}{\vec{e^1_z} \cdot \vec{e^2_z}}=tan(\phi)
\end{equation}
which is valid for angles in between $-\frac{\pi}{2}< \theta < \frac{\pi}{2}$ and 
the constraint on twisting motion is $\phi=0$.
\begin{thebibliography}{99}

\bibitem{Erleben}
Erleben, Sporring, Henriksen and Dohlman \textit{PHYSICS-BASED ANIMATION, Chapter 7},
(THOMOSON, 2005).

\bibitem{IEEE}
Martin Servin and Claude Lacoursiére \textit{Rigid Body Cable for Virtual 
Environments, the angular constraint},
(IEEE TRANSACTIONS ON VISUALIZATION AND COMPUTER GRAPHICS, 2007).

\end{thebibliography}


\end{document}
